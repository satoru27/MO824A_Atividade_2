\documentclass{TEMA}
\usepackage[utf8]{inputenc}
\usepackage[brazil]{babel}      % para texto em Português
\usepackage{amsfonts}
\usepackage{amsmath,amssymb}
\usepackage{graphicx}
\usepackage{xcolor}
\newcommand{\Z}{\mathbb{Z}}

\title{Atividade 2 - Grupo 8 - MO824}

\author{
    Victor Ferreira Ferrari      - RA 187890\\
    Flávio Murilo Reginato       - RA 197088\\
    Vitor Satoru Machi Matsumine - RA 264962\\
}

\date{02/10/2020}

\begin{document}

\criartitulo

\runningheads{V.F. Ferrari, F.M. Reginato e V.S.M. Matsumine}%
{Atividade 2}


\begin{abstract}

{\bf Resumo}. Este documento propõe uma solução para o problema 2-TSP, apresentando uma modelagem matemática para o problema linear, realizando ensaios através do software Gurobi. Essa solução é então avaliada em comparação com a solução do problema TSP. Todas as instâncias testadas foram resolvidas de maneira ótima, utilizando \textit{lazy constraints}. O custo da solução do TSP multiplicado por 2 é um limitante inferior para o custo da solução do 2-TSP para a mesma instância, porém a qualidade desse limitante pode diminuir a utilidade (\textit{gap} de em média 19\%).

{\bf Palavras-chave}. Programação Linear, Otimização, Gurobi, 2-TSP.

\end{abstract}

\section{Introdução}

    Este trabalho consiste na apresentação de um modelo matemático e na realização de experimentações para o problema proposto na Atividade 2 de MO824 2S-2020.
    Chamado 2-TSP, o problema é uma variação do tradicional TSP (\textit{travelling salesman problem}) e busca encontrar dois ciclos hamiltonianos com conjunto de arestas disjuntos e cuja soma de peso dessas arestas seja mínima.
    
    Para isso, foi empregada programação linear utilizando \textit{lazy constraints} através do software Gurobi, onde o problema foi modelado baseando-se na implementação que a própria documentação do software fornece para o TSP.

\section{Modelo Matemático}
    As seguintes variáveis foram usadas no modelo:
    
    $x_{e}$: Variável de decisão binária associada à presença ($x_e = 1$) ou não ($x_e = 0$) da aresta $e$ no primeiro ciclo.
    
    $y_{e}$: Variável de decisão binária associada à presença ($y_e = 1$) ou não ($y_e = 0$) da aresta $e$ no segundo ciclo.\\
    
\begin{equation*}
\begin{array}{ll@{}ll@{}ll}
\text{min}  & \displaystyle\sum\limits_{e \in E} c_{e}x_{e} + c_{e}y_{e} & &(1)\\
\text{s.a}       & \displaystyle\sum\limits_{e \in \delta(i)} x_{e} = 2 & \forall i \in V &(2)\\
                 & \displaystyle\sum\limits_{e \in \delta(i)} y_{e} = 2 & \forall i \in V &(3)\\
                 & \displaystyle\sum\limits_{e \in E(S)} x_{e} \leq |S|-1 & \forall S \subset V &(4)\\
                 & \displaystyle\sum\limits_{e \in E(S)} y_{e} \leq |S|-1 & \forall S \subset V &(5)\\
                 & x_e + y_e \leq 1 & \forall e \in E &(6)\\
                 & x_{e}, y_{e} \in \{0,1\} & \forall e \in E &(7)\\
\end{array}
\end{equation*}

Onde $\delta(i)$ é o conjunto de arestas incidentes no vértice $i$, $S \subset V$ é um subconjunto próprio de vértices e $E(S)$ é o conjunto das arestas cujas extremidades estão em $S$.\\

O modelo é uma generalização do programa linear inteiro correspondente ao TSP, mas difere-se dele pelo objetivo, que busca minimizar o custo de dois ciclos hamiltonianos disjuntos (equação $1$), e também pelas seguintes restrições: 

\textit{Restrições (2-3)}: garante que todos os vértices dos ciclos tenham apenas duas arestas incidentes.

\textit{Restrições (4-5)}: elimina subciclos ilegais nos ciclos. Podem ser implementadas por \textit{lazy constraints}, já que o número de restrições é exponencial.

\textit{Restrição (6)}: assegura que os conjuntos de arestas dos dois ciclos serão disjuntos.

Esse modelo introduz um novo conjunto de variáveis para o segundo ciclo. Porém, é possível reduzir para um único conjunto de variáveis, com uma dimensão extra. Desse modo, o modelo é facilmente escalável para encontrar $k$ ciclos (\textit{k-TSP}). O modelo foi feito desta maneira por legibilidade.

\section{Metodologia}
\subsection{Gerador de Instâncias}
        As instâncias foram geradas de maneira aleatória e uniforme a partir de uma quantidade de cidades fornecida e uma \textit{random seed}. Como proposto, as cidades foram posicionadas aleatoriamente no plano, e as distâncias euclideanas entre os pares foram calculadas. A geração é feita no mesmo arquivo que a execução, e todas as instâncias foram executadas com a mesma \textit{seed}.
\subsection{Especificações do Computador}
    As especificações de hardware do computador no qual foram feitas as execuções estão na \tablename~\ref{table:conditions}.
    \begin{table}[ht]
        \centering
        \caption{Condições de Execução}
        \label{table:conditions}
        \vspace{0.3cm}
        \resizebox{70ex}{!}{
        \begin{tabular}{|c|c|}
        \hline
            Modelo da CPU & Intel(R) Core(TM) i7-9750H (6C/12T)\\ \hline
            Frequência do Clock da CPU & 2.60 GHz\\ \hline
            RAM & 16 GB/2660 MHz \\
        \hline
        \end{tabular}
        }
    \end{table}
    
    O sistema operacional utilizado foi o Windows 10 (64 bits). Foi utilizado como software de execução o \textit{solver} Gurobi Optimizer V9.0.3 empregando 12 threads. Os modelos foram executados com limite de 1800 segundos (30 minutos) e sem limite de memória.
\vspace{-0.18cm}
\section{Resultados Obtidos e Análise}
    \paragraph*{}
        Para cada tamanho proposto na atividade foi gerada uma instância e os resultados alcançados estão na \tablename~\ref{table:resultados}.
    
        \begin{table}[ht]
        \caption{Resultados Obtidos}
        \label{table:resultados}
        \vspace{0.3cm}
        \resizebox{\textwidth}{!}{%
        \begin{tabular}{|c|c|c|c|c|c|c|c|}
        \hline
          \begin{tabular}[c]{@{}c@{}}Número de \\ cidades\end{tabular} &
          \begin{tabular}[c]{@{}c@{}}Quantidade de\\ variáveis TSP\end{tabular} &
          \begin{tabular}[c]{@{}c@{}}Custo da solução\\ TSP (multiplicado por 2)\end{tabular} &
          \begin{tabular}[c]{@{}c@{}}Tempo de\\ Execução TSP {[}s{]}\end{tabular} &
          \begin{tabular}[c]{@{}c@{}}Quantidade de\\ variáveis 2-TSP\end{tabular} &
          \begin{tabular}[c]{@{}c@{}}Custo da\\ solução 2-TSP\end{tabular} &
          \begin{tabular}[c]{@{}c@{}}Tempo de\\ Execução 2-TSP {[}s{]}\end{tabular} \\ \hline
        20  & 190   & 8.26110   & 0.03290 & 380  & 10.4170 & 0.03989   \\ \hline
        40  & 780   & 11.81582  & 0.16655 & 1560 & 14.0863 & 0.54155   \\ \hline
        60  & 1770  & 13.21266  & 0.16655 & 3540 & 16.0766 & 1.89393   \\ \hline
        80  & 3160  & 14.90658  & 0.69813 & 6320 & 18.5450 & 14.00453  \\ \hline
        100 & 4950  & 16.07044  & 1.07711 & 9900 & 20.2082 & 20.55900  \\ \hline
        \end{tabular}%
        }
        \end{table}
    
         Pela \tablename~\ref{table:resultados}, temos uma comprovação do comportamento esperado. O custo de uma solução do 2-TSP é no mínimo o dobro do custo da solução do TSP para a mesma instância, no caso em que ambos ciclos têm o mesmo custo, e esse custo é o mínimo. Na maioria dos casos, assim como em todas as instâncias testadas, isso não ocorreu, então o custo do 2-TSP foi maior. 
         
         Além disso, os tempos de execução das instâncias do TSP foram menores que os do 2-TSP. Isso era esperado, porém o aumento dos tempos no 2-TSP seguiu uma curva consideravelmente mais inclinada que no problema simples, indicando assim que o problema tem uma dificuldade maior.
         
         Pelas relações entre os problemas discutidas nesta seção, pode-se concluir que o custo do TSP multiplicado por 2 pode ser utilizado como \textbf{limitante inferior} para o 2-TSP. Para instâncias maiores, o tempo necessário para o problema simples é suficientemente pequeno para ser possivelmente vantajoso como limitante inferior inicial. A principal desvantagem é na qualidade desse limitante, que pode variar. O \textit{gap} entre as soluções foi de em média 19\%, suficientemente grande para diminuir a utilidade.

\end{document}